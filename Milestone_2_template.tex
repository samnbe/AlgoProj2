\documentclass{article}
\usepackage{algorithm}
\usepackage{algpseudocode}
\usepackage{amsmath}
\usepackage{mathtools}
\usepackage{float}
\usepackage{graphicx}
\usepackage{pgfplots}
\usepackage{hyperref}

\pgfplotsset{compat=1.14}
\begin{document}

\title{COP4533 - Programming Assignment Milestone-2 Report}
\date{}
\maketitle

\section{Team Members}
Samantha Bennett worked on creating and implementing each algorithm and their proofs. Sophia Shah worked on creating each algorithm, the comparative study, and the algorithm analysis.

\section{Algorithm Design and Analysis}

\subsection{Algorithm 3}
\subsubsection{Description}

We can solve the collecting treasure problem using a recursive approach \textbf{without} memoization.

\textbf{Recursive strategy:} \\
At every vault index $i$, the algorithm computes the maximum value that can be collected by considering two options:
\begin{enumerate}
    \item \textbf{Include} the current vault $i$, and skip the next $k$ vaults by calling the recursive function on index $i - k - 1$.
    \item \textbf{Exclude} the current vault $i$, and instead call the recursive function on index $i - 1$.
\end{enumerate}
The recursive function then returns the maximum of these two values.

\textbf{Step-by-step example:} \\
Suppose we have the list of treasure vault values and $k = 2$:
\[
[1, 2, 3, 4, 5, 6, 7].
\]
\begin{enumerate}
    \item \textbf{Base case:}
    \[
    OPT[0] = 0.
    \]
    When there are no vaults, the maximum value is $0$.

    \item \textbf{For $i = 1$:}
    \[
    OPT[1] = \max(values[1] + OPT[-2],\; OPT[0]) = \max(1 + 0,\; 0) = 1.
    \]
    So we take vault $1$.

    \item \textbf{For $i = 2$:}
    \[
    OPT[2] = \max(values[2] + OPT[-1],\; OPT[1]) = \max(2 + 0,\; 1) = 2.
    \]
    Vault $2$ is better than vault $1$ alone.

    \item \textbf{For $i = 3$:}
    \[
    OPT[3] = \max(values[3] + OPT[0],\; OPT[2]) = \max(3 + 0,\; 2) = 3.
    \]
    We take vault $3$.

    \item \textbf{For $i = 4$:}
    \[
    OPT[4] = \max(values[4] + OPT[1],\; OPT[3]) = \max(4 + 1,\; 3) = 5.
    \]
    The best combination up to vault $4$ is vaults $1$ and $4$.

    \item \textbf{For $i = 5$:}
    \[
    OPT[5] = \max(values[5] + OPT[2],\; OPT[4]) = \max(5 + 2,\; 5) = 7.
    \]
    The optimal set up to vault $5$ is vaults $2$ and $5$.

    \item \textbf{For $i = 6$:}
    \[
    OPT[6] = \max(values[6] + OPT[3],\; OPT[5]) = \max(6 + 3,\; 7) = 9.
    \]
    The optimal set up to vault $6$ is vaults $3$ and $6$.

    \item \textbf{For $i = 7$:}
    \[
    OPT[7] = \max(values[7] + OPT[4],\; OPT[6]) = \max(7 + 5,\; 9) = 12.
    \]
    The optimal set up to vault $7$ is vaults $1$, $4$, and $7$.

\end{enumerate}

\textbf{Result:} \\
The final maximum total value is:
\[
OPT[7] = 12.
\]
By backtracking through the computed values, the selected vaults are:
\[
[1, 4, 7].
\]
Hence, the total value collected is $1 + 4 + 7 = 12$.

\vspace{1em}

This algorithm does \textbf{not} use memoization. Therefore, the same subproblems are recomputed many times, leading to an exponential running time.

\begin{algorithm}[H]
\renewcommand{\thealgorithm}{3}
\caption{Finding Maximum Total Vault Value (Recursive without Memoization)}\label{alg:recursive}
\begin{algorithmic}[3]
\Procedure{MaxVault}{$n, k, v_1, v_2, \ldots, v_n$}
    \State $OPT \gets [-1]$ repeated $n + 1$ times
    \State $S \gets \emptyset$ \Comment{set of selected vault indices}
    \Function{ComputeOPT}{$i$}
        \If{$i \leq 0$}
            \State \Return 0
        \EndIf
        \State $include \gets v_i +$ \Call{ComputeOPT}{$i - k - 1$}
        \State $exclude \gets$ \Call{ComputeOPT}{$i - 1$}
        \State $OPT[i] \gets \max(include, exclude)$
        \State \Return $OPT[i]$
    \EndFunction
    \State $TotalValue \gets$ \Call{ComputeOPT}{$n$}
    \State $i \gets n$
    \While{$i > 0$}
        \If{$OPT[i] \neq OPT[i - 1]$}
            \State $S \gets S \cup \{i\}$
            \State $i \gets i - k - 1$
        \Else
            \State $i \gets i - 1$
        \EndIf
    \EndWhile
    \State \Return (TotalValue, $S$)
\EndProcedure
\end{algorithmic}
\end{algorithm}


\subsubsection{Correctness Proof}
{\color{red} Provide a clear argument or formal proof that your algorithm always gives the correct solution for ProblemG.}

\subsubsection{Runtime Analysis}

Let $n$ be the number of vault values in the list.  
The recursive algorithm evaluates every possible combination of vault selections under the $k$-spacing constraint.  
At each index $i$, the algorithm makes two recursive calls:
\begin{enumerate}
    \item One call that \textbf{includes} the current vault $i$ and then proceeds to index $(i - k - 1)$.
    \item One call that \textbf{excludes} the current vault $i$ and proceeds to index $(i - 1)$.
\end{enumerate}

This results in a binary branching structure of recursive calls.  
Let $T(n)$ denote the number of operations required to compute the result for $n$ vaults.  
The recurrence relation can be expressed as:

\[
T(n) = T(n - 1) + T(n - k - 1) + c,
\]
where $c$ is the constant time required for the comparison and addition at each step.

In the worst case, when $k$ is small ($k = 0$ or $k = 1$), the recurrence simplifies to approximately:

\[
T(n) = T(n - 1) + T(n - 2) + c.
\]

Therefore, the closed-form growth of $T(n)$ is exponential:

\[
T(n) = O(2^n).
\]

Even for larger $k$, the recursive tree still exhibits exponential growth because the number of unique subproblems increases exponentially without memoization.  
Each call recomputes results for overlapping subproblems many times.

Thus, the overall running time of the recursive vault-selection algorithm \textbf{without memoization} is:

\[
T(n, k) = O(2^n).
\]

If memoization or dynamic programming were applied, the number of unique subproblems would be reduced to $n$, resulting in a polynomial runtime.  
However, in this unoptimized recursive version, every subproblem is recalculated repeatedly, leading to exponential time complexity.


\subsection{Algorithm 4}
\subsubsection{Description}

We can solve the collecting treasure problem using dynamic programming.  
In this approach, each subproblem corresponds to finding the maximum total vault value obtainable from the first $i$ vaults under the $k$-spacing constraint.  

The recursive relation for the problem is defined as:

\[
OPT[i] =
\begin{cases}
0, & \text{if } i = 0, \\[6pt]
\displaystyle \max_{1 \leq j \leq i} \{ v_j + OPT[j - k - 1] \}, & \text{otherwise.}
\end{cases}
\]

Intuitively, for each position $i$, we examine every vault $j$ up to $i$ and consider including vault $j$ in the optimal subset.  
If vault $j$ is chosen, all vaults from $(j - k)$ through $(j - 1)$ must be excluded, and the best total including $v_j$ becomes $v_j + OPT[j - k - 1]$.  
The maximum of all such choices for $j$ yields $OPT[i]$.

\vspace{1em}

\textbf{Algorithm 4A (Top-Down Recursive DP with Memoization):} \\
Algorithm 4A implements this recurrence recursively.  
The function computes $OPT[i]$ using memoization to store previously computed values in the array \texttt{OPT}.  
Whenever a new subproblem is encountered, it explores all $j$ from $1$ to $i$, updating the maximum total value.  
Once computed, $OPT[i]$ is stored to avoid redundant recursive calls.  
After computing $OPT[n]$, the algorithm backtracks to reconstruct the chosen vault indices.

\begin{algorithm}[H]
\renewcommand{\thealgorithm}{4A}
\caption{Finding Maximum Total Vault Value (Top-Down Recursive DP with Memoization)}\label{alg:topdown}
\begin{algorithmic}[1]
\Procedure{MaxVault-TopDown}{$n, k, v_1, v_2, \ldots, v_n$}
    \State $OPT \gets [-1]$ repeated $n + 1$ times
    \State $S \gets \emptyset$ \Comment{set of selected vault indices}
    \Function{ComputeOPT}{$i$}
        \If{$i \leq 0$}
            \State \Return 0
        \EndIf
        \If{$OPT[i] \neq -1$}
            \State \Return $OPT[i]$
        \EndIf
        \State $exclude \gets$ \Call{ComputeOPT}{$i - 1$}
        \State $best \gets exclude$
        \For{$j = 1$ to $i$}
            \State $include \gets v_j$
            \If{$j - k - 1 > 0$}
                \State $include \gets include +$ \Call{ComputeOPT}{$j - k - 1$}
            \EndIf
            \State $best \gets \max(best, include)$
        \EndFor
        \State $OPT[i] \gets best$
        \State \Return $OPT[i]$
    \EndFunction
    \State $TotalValue \gets$ \Call{ComputeOPT}{$n$}
    \State $i \gets n$
    \While{$i > 0$}
        \If{$OPT[i] \neq OPT[i - 1]$}
            \State $S \gets S \cup \{i\}$
            \State $i \gets i - k - 1$
        \Else
            \State $i \gets i - 1$
        \EndIf
    \EndWhile
    \State \Return ($TotalValue$, $S$)
\EndProcedure
\end{algorithmic}
\end{algorithm}

\vspace{1em}

\textbf{Step-by-step example:} \\
Suppose we have the list of vault values and $k = 2$:
\[
[1, 2, 3, 4, 5, 6, 7].
\]

We use the recurrence relation:
\[
OPT[i] =
\begin{cases}
0, & \text{if } i = 0, \\[6pt]
\displaystyle \max_{1 \leq j \leq i} \{ v_j + OPT[j - k - 1] \}, & \text{otherwise.}
\end{cases}
\]

\textbf{Step-by-step computation:}
\[
k = 2
\]

\begin{enumerate}
    \item $OPT[0] = 0$
    \item $OPT[1] = \max(v_1) = 1$
    \item $OPT[2] = \max(v_1, v_2) = 2$
    \item $OPT[3] = \max(v_1, v_2, v_3 + OPT[0]) = \max(1, 2, 3 + 0) = 3$
    \item $OPT[4] = \max(v_1, v_2, v_3, v_4 + OPT[1]) = \max(1, 2, 3, 4 + 1) = 5$
    \item $OPT[5] = \max(v_1, v_2, v_3, v_4, v_5 + OPT[2]) = \max(1, 2, 3, 4, 5 + 2) = 7$
    \item $OPT[6] = \max(v_1, \dots, v_6 + OPT[3]) = \max(1, 2, 3, 4, 5, 6 + 3) = 9$
    \item $OPT[7] = \max(v_1, \dots, v_7 + OPT[4]) = \max(1, 2, 3, 4, 5, 6, 7 + 5) = 12$
\end{enumerate}

At the end of the recursive computation:
\[
OPT = [0, 1, 2, 3, 5, 7, 9, 12].
\]

\textbf{Backtracking:}
\begin{itemize}
    \item Start at $i = 7$. Since $OPT[7] \neq OPT[6]$, select vault $7$.
    \item Jump $k + 1 = 3$ steps back $\Rightarrow i = 4$.
    \item $OPT[4] \neq OPT[3]$, select vault $4$.
    \item Jump back to $i = 1$.
    \item $OPT[1] \neq OPT[0]$, select vault $1$.
\end{itemize}

The selected vaults are:
\[
[1, 4, 7],
\]
with a total value of:
\[
1 + 4 + 7 = 12.
\]

\bigskip
\hrule
\bigskip

\vspace{1em}

\textbf{Algorithm 4B (Bottom-Up Iterative DP):} \\
Algorithm 4B uses an iterative version of the same recurrence.  
It builds the \texttt{OPT} table from the base case upward:
\[
OPT[0] = 0, \qquad
OPT[i] = \max_{1 \leq j \leq i} \{ OPT[i - 1],\; v_j + OPT[j - k - 1] \}.
\]
Each $OPT[i]$ represents the best possible total up to vault $i$.  
Once the table is filled, the algorithm backtracks from $i = n$ to reconstruct the chosen vaults.

\begin{algorithm}[H]
\renewcommand{\thealgorithm}{4B}
\caption{Finding Maximum Total Vault Value (Bottom-Up Iterative DP)}\label{alg:bottomup}
\begin{algorithmic}[1]
\Procedure{MaxVault-BottomUp}{$n, k, v_1, v_2, \ldots, v_n$}
    \State $OPT[0] \gets 0$
    \State $S \gets \emptyset$
    \For{$i = 1$ to $n$}
        \State $OPT[i] \gets OPT[i - 1]$
        \For{$j = 1$ to $i$}
            \State $include \gets v_j$
            \If{$j - k - 1 > 0$}
                \State $include \gets include + OPT[j - k - 1]$
            \EndIf
            \State $OPT[i] \gets \max(OPT[i], include)$
        \EndFor
    \EndFor
    \State $i \gets n$
    \While{$i > 0$}
        \If{$OPT[i] \neq OPT[i - 1]$}
            \State $S \gets S \cup \{i\}$
            \State $i \gets i - k - 1$
        \Else
            \State $i \gets i - 1$
        \EndIf
    \EndWhile
    \State \Return ($OPT[n]$, $S$)
\EndProcedure
\end{algorithmic}
\end{algorithm}

\textbf{Step-by-step example:} \\
Suppose we have the list of vault values and $k = 2$:
\[
[1, 2, 3, 4, 5, 6, 7].
\]

We use the same recurrence relation but fill the $OPT$ table iteratively:
\[
OPT[0] = 0, \qquad
OPT[i] = \max_{1 \leq j \leq i} \{ OPT[i - 1],\; v_j + OPT[j - k - 1] \}.
\]

\textbf{Step-by-step computation:}

\begin{enumerate}
    \item $OPT[0] = 0$
    \item $i = 1$: $include = 1$, so $OPT[1] = 1$
    \item $i = 2$: $include = 2$, so $OPT[2] = 2$
    \item $i = 3$: $include = 3 + OPT[0] = 3$, so $OPT[3] = \max(2, 3) = 3$
    \item $i = 4$: $include = 4 + OPT[1] = 4 + 1 = 5$, so $OPT[4] = \max(3, 5) = 5$
    \item $i = 5$: $include = 5 + OPT[2] = 5 + 2 = 7$, so $OPT[5] = \max(5, 7) = 7$
    \item $i = 6$: $include = 6 + OPT[3] = 6 + 3 = 9$, so $OPT[6] = \max(7, 9) = 9$
    \item $i = 7$: $include = 7 + OPT[4] = 7 + 5 = 12$, so $OPT[7] = \max(9, 12) = 12$
\end{enumerate}

At the end of the loop:
\[
OPT = [0, 1, 2, 3, 5, 7, 9, 12].
\]

\textbf{Backtracking:}
\begin{itemize}
    \item Start at $i = 7$. Since $OPT[7] \neq OPT[6]$, select vault $7$.
    \item Jump $k + 1 = 3$ steps back $\Rightarrow i = 4$.
    \item $OPT[4] \neq OPT[3]$, select vault $4$.
    \item Jump back to $i = 1$.
    \item $OPT[1] \neq OPT[0]$, select vault $1$.
\end{itemize}

The selected vaults are:
\[
[1, 4, 7],
\]
with a total value of:
\[
1 + 4 + 7 = 12.
\]

\bigskip
\hrule
\bigskip

\subsubsection{Correctness Proof}

{\color{red} Provide a clear argument or formal proof that your algorithm always gives the correct solution for ProblemG.\\

Example using Knapsack problem:\\  
}

\noindent\textbf{Theorem.}  
The DP formulation correctly computes the optimal solution for Knapsack problem

\noindent\textbf{Proof.}\\

We show that the Knapsack problem satisfies the optimal substructure property,  
and that the recursive formulation correctly expresses it.

Let $O_n$ be an optimal subset of items for a knapsack of capacity $W$ and $n$ total items.

\textbf{Case 1:} Item $n$ is not included in $O_n$.  
Then $O_n$ must also be an optimal subset of the first $n-1$ items with the same capacity $W$.  
If there were a subset $Q$ of the first $n-1$ items with higher total value,  
then $Q$ would also be a better solution for the full problem, contradicting the optimality of $O_n$.

\textbf{Case 2:} Item $n$ is included in $O_n$.  
Then the remaining items $O_n - \{n\}$ form an optimal subset of the first $n-1$ items  
with capacity $W - w_n$.  
If $O_n - \{n\}$ were not optimal for that subproblem,  
there would exist another subset $Q$ with higher total value and total weight not exceeding $W - w_n$.  
Adding item $n$ to $Q$ would give a feasible solution for the full problem with higher value than $O_n$,  
contradicting its optimality.

From these two cases, any optimal solution for $(n, W)$ must come from one of two optimal subproblems:
\[
\text{either } (n-1, W) \text{ (excluding item $n$)} \quad \text{or} \quad (n-1, W - w_n) \text{ (including item $n$)}.
\]

Hence, the Knapsack problem satisfies the optimal substructure property,  
and the recursive formulation correctly captures this relationship.



\subsubsection{Runtime Analysis}

Let $n$ be the number of vaults.

\textbf{Algorithm 4A (Top-Down Recursive DP with Memoization):} \\
Each state $OPT[i]$ is computed at most once because of memoization.  
For each $i$, the algorithm loops over all possible $j$ from $1$ to $i$ to compute
\[
OPT[i] = \max_{1 \leq j \leq i} \{ v_j + OPT[j - k - 1] \}.
\]
The inner loop therefore runs in $O(i)$ time for each $i$, and since $i$ ranges from $1$ to $n$, the total time is:

\[
T(n) = \sum_{i=1}^{n} O(i) = O(n^2).
\]

Memoization ensures that no subproblem is recomputed, so the total number of function calls is $O(n)$, and the dominant cost arises from the nested loop over $j$.

\vspace{1em}

\textbf{Algorithm 4B (Bottom-Up Iterative DP):} \\
The bottom-up version explicitly computes all $n$ states iteratively using the same nested structure.  
For each $i$ from $1$ to $n$, it iterates through all $j$ from $1$ to $i$, resulting in the same cost:

\[
T(n) = \sum_{i=1}^{n} O(i) = O(n^2).
\]

Thus, both algorithms have:

\[
T(n) = \Theta(n^2)
\]

In practice, Algorithm 4B would perform faster because of sequential memory access and the absence of a recursive overhead while Algorithm 4A provides a more intuitive top-down structure with memoization.


\subsection{Algorithm 5}
\subsubsection{Description}

We can solve the collecting treasure problem efficiently using a dynamic programming approach similar to the optimal weighted interval scheduling problem.

The recurrence relation for this algorithm is defined as:
\[
OPT[i] =
\begin{cases}
0, & \text{if } i = 0, \\[6pt]
\max \{ OPT[i - 1],\; v_i + OPT[i - k - 1] \}, & \text{otherwise.}
\end{cases}
\]

For each vault index $i$, the algorithm compares two possible choices:
\begin{enumerate}
    \item \textbf{Exclude} vault $i$: take the optimal value up to the previous vault, $OPT[i - 1]$.
    \item \textbf{Include} vault $i$: take the vault’s value $v_i$ plus the best value achievable $k$ vaults earlier, $OPT[i - k - 1]$.
\end{enumerate}

If $i - k - 1 \leq 0$, the algorithm only compares $v_i$ and $OPT[i - 1]$.  
Each $OPT[i]$ thus stores the maximum achievable total value considering vaults $1$ through $i$.  

Once the table is filled, the algorithm backtracks from the end of the array to reconstruct the indices of the chosen vaults.  
Whenever the maximum value changes between $OPT[i]$ and $OPT[i - 1]$, vault $i$ is included, and the index jumps $k + 1$ steps backward.

\begin{algorithm}[H]
\renewcommand{\thealgorithm}{5}
\caption{Finding Maximum Total Vault Value (Iterative DP)}\label{alg:5}
\begin{algorithmic}[1]
\Procedure{MaxVault}{$n, k, v_1, v_2, \ldots, v_n$}
    \State $OPT[0] \gets 0$
    \State $S \gets \emptyset$ \Comment{set of selected vault indices}
    \For{$i = 1$ to $n$}
        \State $include \gets v_i$
        \If{$i - k - 1 > 0$}
            \State $include \gets include + OPT[i - k - 1]$
        \EndIf
        \State $OPT[i] \gets \max(OPT[i - 1],\; include)$
    \EndFor
    \State $i \gets n$
    \While{$i > 0$}
        \If{$OPT[i] \neq OPT[i - 1]$}
            \State $S \gets S \cup \{i\}$
            \State $i \gets i - k - 1$
        \Else
            \State $i \gets i - 1$
        \EndIf
    \EndWhile
    \State \Return ($OPT[n]$, $S$)
\EndProcedure
\end{algorithmic}
\end{algorithm}
\textbf{Step-by-step example:} \\
Suppose we have the list of vault values and $k = 2$:
\[
[1, 2, 3, 4, 5, 6, 7].
\]

We fill the $OPT$ table iteratively using the recurrence relation.

\begin{enumerate}
    \item $OPT[0] = 0$
    \item $i = 1$: $include = 1$, so $OPT[1] = \max(0, 1) = 1$
    \item $i = 2$: $include = 2$, so $OPT[2] = \max(1, 2) = 2$
    \item $i = 3$: $include = 3 + OPT[0] = 3$, so $OPT[3] = \max(2, 3) = 3$
    \item $i = 4$: $include = 4 + OPT[1] = 4 + 1 = 5$, so $OPT[4] = \max(3, 5) = 5$
    \item $i = 5$: $include = 5 + OPT[2] = 5 + 2 = 7$, so $OPT[5] = \max(5, 7) = 7$
    \item $i = 6$: $include = 6 + OPT[3] = 6 + 3 = 9$, so $OPT[6] = \max(7, 9) = 9$
    \item $i = 7$: $include = 7 + OPT[4] = 7 + 5 = 12$, so $OPT[7] = \max(9, 12) = 12$
\end{enumerate}

At the end of the loop:
\[
OPT = [0, 1, 2, 3, 5, 7, 9, 12].
\]

\textbf{Backtracking:}
\begin{itemize}
    \item Start at $i = 7$. Since $OPT[7] \neq OPT[6]$, select vault $7$.
    \item Jump $k + 1 = 3$ steps back to $i = 4$.
    \item $OPT[4] \neq OPT[3]$, select vault $4$.
    \item Jump back to $i = 1$.
    \item $OPT[1] \neq OPT[0]$, select vault $1$.
\end{itemize}

The selected vaults are:
\[
[1, 4, 7],
\]
with a total value of $1 + 4 + 7 = 12$.


\subsubsection{Correctness Proof}
{\color{red} Provide a clear argument or formal proof that your algorithm always gives the correct solution for ProblemG.}

\subsubsection{Runtime Analysis}

Let $n$ be the number of vaults.  
The algorithm fills the $OPT$ array using a single loop that runs from $1$ to $n$:
\[
\texttt{for i in range(1, n + 1):}
\]
Inside the loop, the algorithm performs a constant number of operations:
one addition, one comparison, and at most one conditional check.

Thus, the time complexity of filling the $OPT$ table is:
\[
T_1(n) = O(n).
\]

The backtracking loop starts from $i = n$ and decrements $i$ by at least $1$ each iteration until reaching $0$.  
Therefore, in the worst case, the backtracking requires at most $n$ steps:
\[
T_2(n) = O(n).
\]

Combining both phases:
\[
T(n) = T_1(n) + T_2(n) = O(n) + O(n) = O(2n) = O(n).
\]

Hence, the overall time complexity is:
\[
T(n) = \Theta(n).
\]

Since each iteration takes constant time and both loops run at most $n$ times, this is the most efficient possible solution for this problem.


\section{Experimental Comparative Study}

{\color{red}The goal of the experimental study is to visualize the growth of the running time of the algorithms as the input size increases. Few things that will make the visualization more accurate.

\begin{itemize}
    \item Generate datasets with size $n$ that goes high enough. Small values of $n$ leads to an inaccurate visualization.
    \item Pick uniformly distributed values of $n$. Non uniformly distributed values of n leads to a misleading visualization.
    \item Use enough samples. Not enough samples of n also leads to an inaccurate visualization.
\end{itemize}
}

\subsection{Experimental Setup}
{\color{red} Describe how you generated random datasets, their sizes, and how you measured running times.

You may include additional plots


}

\subsection{Plot 3}
{\color{red} Insert your plot of running time for Program 3 vs.\ input size. Explain the observed trend.}

\begin{filecontents}{p3.dat}
X   Points   Program3
1   10000    0.016
2   15000    0.024
3   20000    0.030
4   25000    0.036
5   30000    0.043
\end{filecontents}

\begin{figure}[H]
\centering
\begin{tikzpicture}
\begin{axis}[
    xlabel=Number of Elements,
    ylabel=Running Time (seconds),
    xticklabels from table={p3.dat}{Points},
    xtick=data,
    legend style={at={(0.97,0.03)},anchor=south east}
]
\addplot[blue,thick,mark=square*] table [y=Program3,x=X]{p3.dat};
\addlegendentry{Program 3}
\end{axis}
\end{tikzpicture}
\caption{Example plot.}
\label{plot1}
\end{figure}

\subsection{Plot 4}
{\color{red} Insert your plot of running time for Program 4A vs.\ input size. Explain the observed trend.}

\subsection{Plot 5}
{\color{red} Insert your plot of running time for Program 4B vs.\ input size. Explain the observed trend.}

\subsection{Plot 6}
{\color{red} Insert your plot of running time for Program 5 vs.\ input size. Explain the observed trend.}

\subsection{Plot 7}
{\color{red} Overlay Plots 3,4,5,6 and contrasting the performance of Programs 3,4A, 4B,5}

\subsection{Plot 8}
{\color{red} Overlay Plots 4,5 and contrasting the performance of Programs 4A, 4B}

\subsection{Plot 9}
{\color{red} Insert your plot of output quality comparison $(h_g - h_o)/h_o$ of Algorithm1 and any of Algorithms 3,4,5}

\subsection{Observations/Comments}
{\color{red} Inlcude any additional observations or comments.}

\section{Conclusion}
{\color{red} Summarize your learning experience in Milestone 1. Reflect on design, analysis, experiments, and challenges.}

\end{document}